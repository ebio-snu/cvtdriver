\section*{cvtdriver}

cvtdriver 는 \href{https://github.com/ebio-snu/stdcvt}{\tt 스마트팜장비 연동을 위한 컨버터 개발} 프로젝트에서 활용되는 드라이버개발을 돕기 위한 프로젝트이다. 본 프로젝트에서 드라이버 개발을 위한 A\+PI 스펙 및 샘플 드라이버를 제공하여 참여사의 드라이버 제작을 돕는다.

\subsection*{클래스 구조}

드라이버 클래스 구조는 다음과 같다.




\begin{DoxyItemize}
\item \hyperlink{classstdcvt_1_1CvtDriver}{Cvt\+Driver} \+: 드라이버의 인터페이스가 되는 가상 클래스이다.
\item \hyperlink{classstdcvt_1_1CvtOption}{Cvt\+Option} \+: 드라이버 구동을 위한 옵션을 전달하는 클래스이다.
\item \hyperlink{classstdcvt_1_1CvtDeviceSpec}{Cvt\+Device\+Spec} \+: 드라이버에서 다루는 장비의 스펙을 다루는 클래스이다. 장비종류, 설치위치, 장비의 작동대상, 제조사, 모델명을 처리한다.
\item \hyperlink{classstdcvt_1_1CvtDevice}{Cvt\+Device} \+: 드라이버에서 다루는 장비의 인터페이스가 되는 가상 클래스이다.
\begin{DoxyItemize}
\item \hyperlink{classstdcvt_1_1CvtSensor}{Cvt\+Sensor} \+: 센서의 인터페이스가 되는 클래스이다. 간단한 경우라면 직접 사용이 가능하다.
\item \hyperlink{classstdcvt_1_1CvtMotor}{Cvt\+Motor} \+: 모터형 구동기의 인터페이스가 되는 클래스이다. 간단한 경우라면 직접 사용이 가능하다.
\item Cvt\+Switch \+: 스위치형 구동기의 경우 간단하여 구현되어 있지 않다. Cvt\+Device를 그냥 사용한다.
\end{DoxyItemize}
\item \hyperlink{classstdcvt_1_1CvtCommand}{Cvt\+Command} \+: 구동기에서 처리하는 명령의 인터페이스가 되는 클래스이다.
\begin{DoxyItemize}
\item \hyperlink{classstdcvt_1_1CvtRatioCommand}{Cvt\+Ratio\+Command} \+: 구동기 전달하는 명령이 비율(퍼센트)인 경우에 사용하는 클래스이다. 모터형 구동기에 적합한 명령클래스라고 할 수 있다.
\end{DoxyItemize}
\end{DoxyItemize}

\subsection*{샘플 드라이버}

두가지 종류의 샘플드라이버를 제공한다.


\begin{DoxyItemize}
\item \hyperlink{classebiodriver_1_1DSSampleDriver}{D\+S\+Sample\+Driver} \+: 별도로 제공되는 샘플 노드와 serial 통신을 수행하는 드라이버이다. boost\+::asio 를 사용한다.
\item \hyperlink{classebiodriver_1_1SSSampleDriver}{S\+S\+Sample\+Driver} \+: 별도로 제공되는 테스트\+U\+I와 연동되는 드라이버이다. (현재는 구현이 안되어있다.) 
\end{DoxyItemize}