드라이버는 크게 2종류로 구분된다. 센서노드, 제어노드, 컨트롤러와 직접 연결되는 장비측(\+Device Side)드라이버와 데이터 수집기와 연결되는 서버측(\+Servier Side)드라이버로 구성된다. 각 드라이버를 D\+S\+Driver, S\+S\+Driver 라고 한다.

드라이버 A\+P\+I는 기초연동용과 고급연동용으로 구분될 계획이다. 현 단계에서 드라이버는 현상황을 충실히 포함하는쪽으로 설계하는 것을 목표로 하고 있다. 추후에 활성화가 되어 추상화 레벨을 높여서 더 많은 범위를 커버할 수 있기를 바란다.

\subsection*{드라이버 스펙 문서}

현 시점에서 기초 연동을 위한 A\+PI 초안이 개발되어 있으며, 참여사와의 워크샵을 통해 기초 연동용 최종안이 개발될 것으로 기대된다.

드라이버 개발을 위해서 \href{https://ebio-snu.github.io/cvtdriver/}{\tt Cvt\+Driver 레퍼런스}를 참고한다.

\subsection*{Device ID, Device Spec, Device Index}

컨버터와 드라이버의 이해를 위해서 ID, Spec, Index를 구분하는 것이 필요하다.

Device ID 는 개발사에서 관리하는 I\+D이며, 다양한 방식의 아이디를 고려하여 문자열 형이다. 컨버터에서는 I\+D를 기반한 작업이 없지만, Cvt\+Raw\+Command를 사용해서 특정 I\+D를 가진 장비에게 명령을 전송할 수는 있다.

컨버터에서는 Index와 Spec에는 관심이 많다. Index는 컨버터에서 I\+D와 같이 사용되는 값이다. 드라이버는 getdevice 메소드로 장비를 요청할 때 동일한 index에 대해서 동일한 장비를 리턴해야 한다.

Device\+Spec은 서로 다른 드라이버 사이에서 정보를 교환할때 기준이 되는 값이다. 서로 다른 드라이버는 서로 다른 개발사에서 개발이 되기 때문에 서로 I\+D를 공유할 수 없을 뿐만아니라 해당 장비의 종류, 설치위치 등을 알 수 없다. 각 개발사에서는 해당 장비의 Spec을 정확하게 기록해야 한다.

특정 장비의 Spec을 공통코드로 표현할 수 없을때에는 즉각 이를 알려서 공통코드를 추가하거나 변경할 수 있도록 하여야 한다. 경우에 따라 공통코드의 대대적인 변경이 필요할 경우에는 U\+N\+K\+N\+O\+W\+N을 활용하도록 한다.

\subsection*{드라이버가 가진 장비 데이터의 공유}

2개 이상의 D\+S\+Driver를 사용하는 경우 개별 D\+S\+Driver는 서로 다른 장비를 관리해야한다. 하나의 D\+S\+Driver에서 관리되는 장비에 대한 정보는 다른 장비로 전달될 수 있다.

드라이버로부터 장비 정보를 획득하기 위해서는 Cvt\+Device $\ast$getdevice(int index); 메소드를 활용한다. index는 0부터 시작하고, 값을 하나씩 올려가면서 장비를 꺼낼 수 있다. 이때 리턴값이 nullptr(\+N\+U\+L\+L)인 경우 관리하는 장비가 더이상 없다는 의미이다.

장비의 정보를 다른 드라이버로 전달하기 위해서 bool sharedevice(\+Cvt\+Device $\ast$pdevice); 메소드를 활용한다.

\subsection*{다른 드라이버로 제어명령 전달}

장비에 대한 명령도 다른 드라이버로 전달될 수 있다. 다만 초기버전에서는 S\+S\+Driver가 하나만 존재하고, S\+S\+Driver에서만 명령을 전달할 수 있는 것으로한다.

하나의 드라이버(\+S\+S\+Driver)가 전달하고자 하는 명령은 Cvt\+Command $\ast$getcommand(int index); 메소드를 이용해 얻을 수 있다. index는 0부터 시작하고, 값을 하나씩 올려가면서 명령을 꺼낼 수 있다. 이때 리턴값이 nullptr(\+N\+U\+L\+L)인 경우 더이상의 명령이 없다는 의미이다.

획득한 명령을 다른 드라이버로 전달하기 위해서 bool control(\+Cvt\+Command $\ast$pcmd); 메소드를 활용한다.

\subsection*{드라이버 설정}

드라이버의 설정은 conf/cvtdriver.\+json 에 기록되어 있다. 설정파일은 ssdriver 와 dsdriver 로 나뉘어지는데 기본적인 구조는 동일하다.


\begin{DoxyCode}
1 \{
2     "ssdriver": [\{
3         "driver": "libsssample.so",
4         "option": \{
5             "value": "value.json",
6             "command": "command.json"
7         \}
8     \}],
9     "dsdriver": [\{
10         "driver": "libdssample.so",
11         "option": \{
12             "port": "/dev/ttyUSB0",
13             "baudrate": 115200
14         \}
15     \}]
16 \}
\end{DoxyCode}


하나의 드라이버에 대한 설정은 드라이버 파일명(\char`\"{}driver\char`\"{})과 해당 드라이버 구동을 위한 옵션(\char`\"{}option\char`\"{})으로 구성된다. 드라이버 파일명은 추후 협회의 시스템이 구축될때 내부적인 규칙에 따라 정리될 예정이다. 드라이버의 옵션은 해당드라이버에 맞게 개발자가 설정하면 된다.

위에서 S\+S\+Sample\+Driver 는 파일에 정보를 기록하고, 파일로부터 명령을 읽는 구조를 가지고 있어 해당 파일명들이 옵션에 기록되고, D\+S\+Sample\+Driver는 시리얼 통신을 통해 샘플노드와 통신을 하는 드라이버이기 때문에 옵션으로 port와 baudrate를 가지고 있다. 